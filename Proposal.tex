\documentclass[10pt]{article}
\usepackage{fullpage,listings}
\usepackage{color,xcolor}
\usepackage{caption}

\newcommand{\Lang}{GAMMA}
\newcommand{\Compiler}{ray}

\renewcommand{\lstlistingname}{Example}
\renewcommand{\lstlistlistingname}{Examples}

\setlength{\parskip}{4pt}

\title{\Lang{}: A Strict yet Fair Programming Language}
\author{
	Ben Caimano - blc2129@columbia.edu \\
	Weiyuan Li - wl2453@columbia.edu \\
	Matthew H Maycock - mhm2159@columbia.edu \\
	Arthy Padma Anandhi Sundaram - as4304@columbia.edu
}
\date{}

\begin{document}

%Title area
\maketitle
\begin{center}
\large
A Project for Programming Languages and Translators,
\\taught by Stephen Edwards
\end{center}


\section*{Why \Lang{}? -- The Core Concept}
We propose to implement an elegant yet secure general purpose object-oriented programming language. Interesting features have been selected from the history of object-oriented programming and will be combined with the familiar ideas and style of modern languages.

\Lang{} combines three disparate but equally important tenants:


\begin{enumerate}
\item{Purely object-oriented 
    
    \Lang{} brings to the table a purely object oriented programming language where every type is
    modeled as an object--including the standard primitives. Integers, Strings, Arrays, and other types may be expressed in the standard fashion but are objects behind the scenes and can be treated as such.}

\item{Controllable

   \Lang{} provides innate security by choosing object level access
   control as opposed to class level access specifiers. Private members of one object
   are inaccessible to other objects of the same type. Overloading is not allowed.
   No subclass can turn your functionality on its head.}

\item{Versatile

    \Lang{} allows programmers to place "refinement methods" inside their code.
    Alone these methods do nothing, but may be defined by subclasses so as to extend
    functionality at certain important positions. Anonymous instantiation allows for
    extension of your classes in a quick easy fashion. Generic typing on method
    parameters allows for the same method to cover a variety of input types.}
\end{enumerate}

\section*{ The Motivation Behind \Lang{}}
\Lang{} is a reaction to the object-oriented languages before it.
Obtuse syntax, flaws in security, and awkward implementations plague
the average object-oriented language. \Lang{} is intended as a step
toward ease and comfort as an object-oriented programmer.


The first goal is to make an object-oriented language that is comfortable
in its own skin. It should naturally lend itself to constructing API-layers
and abstracting general models. It should serve the programmer towards their
goal instead of exerting unnecessary effort through verbosity and awkwardness
of structure.


The second goal is to make a language that is stable and controllable.
The programmer in the lowest abstraction layer has control over how those
higher may procede. Unexpected runtime behavior should be reduced through
firmness of semantic structure and debugging should be a straight-forward
process due to pure object and method nature of \Lang{}.

\section*{\Lang{} Feature Set}

\Lang{} will provide the following features:

\begin{itemize}
\item Universal objecthood
\item Optional ``refinement'' functions to extend superclass functionality
\item Anonymous class instantiation
\item Static typing with generic method parameters
\item Access specifiers that respect object boundaries, not class boundaries
\end{itemize}

\section*{\Compiler{}: The \Lang{} Compiler}

The compiler will proceed in two steps. First, the compiler will interpret
the source containing possible syntactic shorthand into a file
consisting only of the most concise and structurally sound GAMMA core. After this the compiler will transform
general patterns into (hopefully portable) C code, and compile this to
machine code with whatever compiler the user specifies.

\section*{Code Example}

\begin{lstlisting}[numbers=left,label=Personhood,caption=Personhood]
Class Person:
  Protected:
    ## instance variables have a type an access class
    String first_name
    String last_name

    ## constructors take arguments, initialize state
    init(String first_name, String last_name):
      this.first_name = first_name
      this.last_name = last_name

  Public:
    ## methods can return values and can be specialized
    ## via refinement
    String toString:
      String result = this.first_name + " " + this.last_name + " is being stringified"
      
      ## Only use this line if we have a defined
      ## refining method
      if refinable(extra):
        result = result + " into a " + refine extra()
      
      result = result + "!"
      
      return result

Class Student extends Person:
  ## subclasses have their own data
  Private:
    String level # Freshman, etc...

  Protected:
    ## Subclasses have to invoke their superclass,
    ## just like in common OOP languages (java/etc)
    init(string first, string last, string grade):
      super(first, last)
      this.level = grade

  Public:
    ## Now we can refine!
    String toString.extra:
      String level_result = this.last_name + ", " + this.first_name + " is a " + this.level
      println(level_result)
      return level_result

## Call our class
Person average = new Person("John", "Smith")
println(average.toString)

println("")

Student very_smart = new Student("Roger", "Penrose", "Graduate Student")
println(very_smart.toString)
  
\end{lstlisting}

\begin{verbatim}
The above program should print:
    John Smith is being stringified!

    Penrose, Roger is a Graduate Student
    Roger Penrose is being stringified into Penrose, Roger is a Graduate Student!
\end{verbatim}

\end{document}