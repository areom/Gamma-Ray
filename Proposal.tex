\documentclass[10pt]{article}
\usepackage{fullpage}

\title{GAMMA: A BETA++ Programming Language}
\author{
	Ben Caimano - blc2129@columbia.edu \\
	Weiyuan Li - wl2453@columbia.edu \\
	Matthew H Maycock - mhm2159@columbia.edu \\
	Arthy Padma Anandhi Sundaram - as4304@columbia.edu
}
\date{}

\begin{document}

%Title area
\maketitle
\begin{center}
\large
A Project for Programming Languages and Translators,
\\taught by Stephen Edwards
\end{center}


\section*{Why GAMMA -- or Why Update BETA?}
BETA is a programming language of the ``Scandanvian School'' of Object
orientation. The language provided two primary beneficial features:

\begin{enumerate}
\item The BETA Pattern is used as a unifying concept for every element
of the program -- Classes, Procedures, anonymous Objects, etc.
\item Refinement vs Overriding is used in BETA to ensure that superclass
behavior is respected in subclass definition. A simple way analogy to
understand this is to think how you would program in Java if all your
methods had to be declared \emph{abstract} or \emph{final}. BETA just
provides this structuring and a little help to make it easier to use.
\end{enumerate}

A consequence of the fact that patterns are refined and everything is a
pattern is that classes can have nested classes and \emph{those} nested
classes can themselves be refined; classes can be \emph{virtual} in the
sense of virtual functions, etc in other languages.

The primary negative about BETA however is that the syntax is horrible
-- absolutely atrocious. While the language provides a nice theoretical
setting for object oriented development given it's pro's, the fact that
the syntax is such a significant con clearly more than negates any of
the benefits. In a world where LISP, the most beautiful language, is
derided for its parentheses, a language like BETA simply cannot succeed
where Python, Ruby, and other easy to read languages exist.

And so this is the goal of the GAMMA programing language project; to
make a new language with a feature set similar to what BETA provides but
with a readable syntax that is user friendly.

\end{document}