\documentclass[10pt]{article}
\usepackage{fullpage}

\title{GAMMA: A BETA++ Programming Language}
\author{
	Ben Caimano - blc2129@columbia.edu \\
	Weiyuan Li - wl2453@columbia.edu \\
	Matthew H Maycock - mhm2159@columbia.edu \\
	Arthy Padma Anandhi Sundaram - as4304@columbia.edu
}
\date{}

\begin{document}

%Title area
\maketitle
\begin{center}
\large
A Project for Programming Languages and Translators,
\\taught by Stephen Edwards
\end{center}


\section*{Why BETA?}
The original Beta was (and, in a sense, still is) an extension of the
concept of object-oriented programming. Procedures and classes become the
almighty pattern, shaker of digital worlds. Or, at the very least, a
powerful tool for object-based design. The original language featured the
much sought-after features of procedure/class abstraction and inheritence.

Why recreate this Scandanavian masterpiece? The original Beta and its
child gBeta have been out of development for several years. In addition,
the syntax features a special pain only tolerable to those who lack souls.
We seek to refresh Beta into BETA - a spiritual successor that profits
from the lessons of modern programming--as well as a pleasant syntax.

\section*{Features}
\begin{itemize}
\item It is written in ascii!
\end{itemize}

\section*{Syntax}
\begin{center}
\begin{tabular}{|p{.1\textwidth}|p{.7\textwidth}|}
\hline
Operator & Purpose\\ \hline
+ & Adding things\\ \hline
\end{tabular}
\end{center}

\section*{Examples}
It can do stuff!

\end{document}