\section{Lexical Elements}
\subsection{Identifiers}
Identifiers are used for the identification of variable,  methods and types. An identifer is a sequence of alphanumeric characters, uppercase and lowercase, and underscores. A variable or method identifier must start with a lowercase alphabetic character. A type identifier must start with an uppercase alphabetic character.

\subsection{Keywords}
The following words are reserved keywords. They may not be used as identifiers:
\begin{center}
\begin{tabular}{cccccc}
\verb!and! & \verb!class! & \verb!else! & \verb!elsif! & \verb!extends! & \verb!false!\\
\verb!if! & \verb!init! & \verb!main! & \verb!nand! & \verb!new! & \verb!nor!\\
\verb!not! & \verb!null! & \verb!or! & \verb!private! & \verb!protected! & \verb!public!\\
\verb!refinable! & \verb!refine! & \verb!refinement! & \verb!return! & \verb!super! & \verb!this!\\
\verb!to! & \verb!true! & \verb!void! & \verb!while! & \verb!xor!\\
\end{tabular}
\end{center}

\subsection{Operators}
There are a large number of (mostly binary) operators:
\begin{center}
\begin{tabular}{ccccccc}
\verb!=! & \verb!=/=! & \verb|<>| & \verb!<! & \verb!<=! & \verb!>! & \verb!>=!\\
\verb!+! & \verb!-! & \verb!*! & \verb!/! & \verb!%! & \verb!^! & \verb!:=!\\
\verb!+=! & \verb!-=! & \verb!*=! & \verb!/=! & \verb!%=! & \verb!%=! \\
\verb!and! & \verb!or! & \verb!not! & \verb!nand! & \verb!nor! & \verb!xor!&\verb!refinable!\\
\end{tabular}
\end{center}

\subsection{Literal Classes}
A literal class is a value that may be expressed in code without the use of the new keyword. These are the fundamental units of program.

\subsubsection{Integer Literals}
An integer literal is a sequence of digits. It may be prefaced by a unary minus symbol. For example:
\begin{itemize}
\item \verb!777!
\item \verb!42!
\item \verb!2!
\item \verb!-999!
\item \verb!0001!
\end{itemize}

\subsubsection{Float Literals}
A float literal is a sequence of digits and exactly one decimal point/period. It must have at least one digit before the decimal point and at least one digit after the decimal point. It may also be prefaced by a unary minus symbol. For example:
\begin{itemize}
\item \verb!1.0!
\item \verb!-0.567!
\item \verb!10000.1!
\item \verb!00004.70000!
\item \verb!12345.6789!
\end{itemize}

\subsubsection{Boolean Literals}
A boolean literal is a single keyword, either \verb!true! or \verb!false!.

\subsubsection{String Literals}
A string literal consists of a sequence of characters enclosed in double quotes. Note that a string literal can have the newline escape sequence within it (see below), but cannot have a newline (line feed), form feed, carriage return, or vertical tab within it; nor can it have the end of file. Please note that the sequence may be of length zero. For example:
\begin{itemize}
\item \verb!"Yellow matter custard"!
\item \verb!""!
\item \verb!"Dripping\n   from a dead"!
\item \verb!"'s 3y3"!
\end{itemize}

The following are the escape sequences available within a string literal; having a backslash followed by a character outside of those below is an error.
\begin{itemize}
\item \verb!\a! - u0007/alert/BEL
\item \verb!\b! - u0008/backspace/BB
\item \verb!\f! - u000c/form feed/FF
\item \verb!\n! - u000a/linefeed/LF
\item \verb!\r! - u000d/carriage return/CR
\item \verb!\t! - u0009/horizontal tab/HT
\item \verb!\v! - u000b/vertical tab/VT
\item \verb!\'! - u0027/single quote
\item \verb!\"! - u0022/double quote
\item \verb!\\! - u005c/backslash
\item \verb!\0! - u0000/null character/NUL 
\end{itemize}

\subsection{Comments}
Comments begin with the sequence \verb!/*! and end with \verb!*/!. Comments nest within each other.  Comments must be closed before the end of file is reached.

\subsection{Separators}
The following characters are used to deliniate various aspects of program organizaiton (function arguments, array indexing, blocks, and expressions):
\begin{center}
\begin{tabular}{cccccccc}
\verb![! & \verb|]| & \verb!(! & \verb!)! & \verb!{! & \verb!}! & \verb!,! & \verb!;!\\
\end{tabular}
\end{center}

\subsection{Whitespace and Noncanonical Gamma}
As written, canonical Gamma ignores whitespace outside of string literals. Gamma code with rigid whitespace (pythonesque Gamma) a la python and without \verb!;!, \verb!{!, or \verb!}! can easily be feed through a preprocessor to become canonical Gamma. Doing so respects the following rules:
\begin{itemize}
\item Tab characters are equivalent to eight spaces
\item Wherever a \verb!{! could be used to start a scope, a \verb!:! can be used instead
\item After starting a scope with \verb!:! the next line with non-whitespace characters determines the indentation level of that scope.
\begin{itemize}
\item If there is no such line (end of file) then the scope is assumed to end and is equivalent to \verb!{}!.
\item If the next such line is indented no more than the line introducing the scope, then the scope is again equivalent to \verb!{}!.
\item If the next line is indented more than the line introducing the scope then this sets the indentation level of the scope -- all statements in this scope \emph{must} be at the same exact level of indentation. The scope continues until end of file or the indentation returns to the indentation level of an outer scope.
\end{itemize}
\item If the line after a statement \emph{not} ending in a colon is indented more than the given line, then it is considered a continuation of that statement.
\item At the end of a statement (either the end of the line starting the statement, or subsequent lines that are indented more as per the rule above), a newline is equivalent to a semicolon \verb!;!.
\item If a scope is explicitly demarcated with \verb!{! and \verb!}! then there can be no whitespace related scoping within that scope -- all inner scoping and statement termination must be handled with \verb!{!, \verb!}!, \verb!;!. 
\end{itemize}

