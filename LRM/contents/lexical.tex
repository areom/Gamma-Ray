\section{Lexical Elements}
\subsection{Identifiers}
Identifiers are used for the identification of variable,  methods and types. An identifer is a sequence of alphanumeric characters, uppercase and lowercase, and underscores. A variable or method identifier must start with a lowercase alphabetic character. A type identifier must start with an uppercase alphabetic character.

\subsection{Keywords}
The following words are reserved keywords. They may not be used as identifiers:
\begin{center}
\begin{tabular}{ccccccc}
and & class & else & elsif & extends & extends & false\\
if & init & nand & new & nor & null & or\\
private & protected & public & refinable & refine & refinement & return\\
super & this & to & true & while & void & xor\\
\end{tabular}
\end{center}

\subsection{Literal Classes}
A literal class is a value that may be expressed in code without the use of the new keyword. These are the fundamental units of program.

\subsubsection{Integer Literals}
An integer literal is a sequence of digits. It may be prefaced by a unary minus symbol. For example:
\begin{itemize}
\item 777
\item 42
\item 2
\item -999
\item 0001
\end{itemize}

\subsubsection{Float Literals}
A float literal is a sequence of digits and exactly one decimal point/period. It must have at least one digit before the decimal point and at least one digit after the decimal point. It may also be prefaced by a unary minus symbol. For example:
\begin{itemize}
\item 1.0
\item -0.567
\item 10000.1
\item 00004.70000
\item 12345.6789
\end{itemize}