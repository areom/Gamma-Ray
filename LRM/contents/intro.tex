\section{Introduction}
\subsection{Why \Lang{}? -- The Core Concept}
We propose to implement an elegant yet secure general purpose object-oriented programming language. Interesting features have been selected from the history of object-oriented programming and will be combined with the familiar ideas and style of modern languages.

\Lang{} combines three disparate but equally important tenants:


\begin{enumerate}
\item Purely object-oriented 
    
    \Lang{} brings to the table a purely object oriented programming language where every type is
    modeled as an object--including the standard primitives. Integers, Strings, Arrays, and other types may be expressed in the standard fashion but are objects behind the scenes and can be treated as such.

\item Controllable

   \Lang{} provides innate security by choosing object level access
   control as opposed to class level access specifiers. Private members of one object
   are inaccessible to other objects of the same type. Overriding is not allowed.
   No subclass can turn your functionality on its head.

\item Versatile

    \Lang{} allows programmers to place "refinement methods" inside their code.
    Alone these methods do nothing, but may be defined by subclasses so as to extend
    functionality at certain important positions. Anonymous instantiation allows for
    extension of your classes in a quick easy fashion.
\end{enumerate}

\subsection{The Motivation Behind \Lang{}}
\Lang{} is a reaction to the object-oriented languages before it.
Obtuse syntax, flaws in security, and awkward implementations plague
the average object-oriented language. \Lang{} is intended as a step
toward ease and comfort as an object-oriented programmer.


The first goal is to make an object-oriented language that is comfortable
in its own skin. It should naturally lend itself to constructing API-layers
and abstracting general models. It should serve the programmer towards their
goal instead of exerting unnecessary effort through verbosity and awkwardness
of structure.


The second goal is to make a language that is stable and controllable.
The programmer in the lowest abstraction layer has control over how those
higher may procede. Unexpected runtime behavior should be reduced through
firmness of semantic structure and debugging should be a straight-forward
process due to pure object and method nature of \Lang{}.
