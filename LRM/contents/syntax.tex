\section{Syntax}
\subsection{Variable Initialization}
Initializing a variable requires a type and a list of identifiers deliminated by commas:

\begin{lstlisting}
type identifier[, identifier[, identifier[...]]];
\end{lstlisting}

If we wanted to initialize variables for the pythagorean theorem, we would do it like so:

\begin{lstlisting}[caption=Variable Initialization for the Pythagorean Theorem]
Float a, b, c;
\end{lstlisting}

\subsection{Variable Assignment}
Assigning an instance to a variable requires an expression and a variable identifier:

\begin{lstlisting}
identifier := expression;
\end{lstlisting}

If we wanted to assign instances of Integer for our pythagorean theorem, we'd do it like so:

\begin{lstlisting}[caption=Variable Assignment for the Pythagorean Theorem]
a := 3;
b := 4;
\end{lstlisting}

\subsection{Method Invocation}
Invoking a method requires at least an identifier for the method of the current context (i.e. implicit \verb!this! receiver). The instance that the method is invoked upon can be provided as a variable. If it is not provided, the method is invoked upon the global state.

\begin{lstlisting}
[expression.]identifier([expression[, expression[...]]])
\end{lstlisting}

Finishing our pythagorean example, we use method invocations and assignment to calculate the length of our third side, c.

\begin{lstlisting}[label=Method Invocation,caption=Method Invocation for the Pythagorean Theorem Using Methods]
c := ((a.power(2)).plus(b.power(2))).power(0.5);
\end{lstlisting}

\subsection{Method Invocation Using Operators}
Alternatively, certain base methods allow for the use of more familiar binary operators in place of a method invocation.

\begin{lstlisting}
expression operator expression
\end{lstlisting}

Using operators has advantages in clarity and succinctness even if the end result is the same.

\begin{lstlisting}[label=Method Invocation,caption=Method Invocation for the Pythagorean Theorem Using Operators]
c := ( a^2 + b^2 )^0.5;
\end{lstlisting}

\subsection{Operator Precidence}
In the previous examples, parentheses were used heavily in a context not directly related to method invocation. Parentheses have one additional function: they modify precedence among operators. Every operator has a precidence in relation to its fellow operators. Operators of higher precidence are enacted first. Please consider the following table for determining precidence:
\begin{table}[h]\footnotesize
\begin{tabular}{ccccccc}
\verb!:=! & \verb!+=! & \verb!-=! & \verb!*=! & \verb!/=! & \verb!%=! & \verb!^=!\\
or & xor & nor &&&&\\
and & nand &&&&&\\
\verb!=! & \verb!<>! & \verb!=/=! &&&&\\
\verb!>! & \verb!<! & \verb!>=! & \verb!<=! &&&\\
\verb!+! & \verb!-! &&&&&\\
\verb!*! & \verb!/! & \verb!%! &&&&\\
not & \verb!^! &&&&&\\
\verb!(! & \verb!)! &&&&&\\
\multicolumn{3}{c}{array dereferencing}&&&&\\
\multicolumn{3}{c}{method invocation}&&&&\\
\end{tabular}
\caption{Operator Precidence}
\end{table}

\subsection{Method Declaration}
A method definition begins with either a return type or Void. The identifier for the function is followed by a pair of parentheses that may enclose the argument variable definitions. There is one type and one identifier for each argument; and they are delimited by commas. Following the parentheses are a pair of braces around the body of the method. In the body, there must be at least one statement but there may be many more. At least one of those statements must be a return statement. Additionally, refinements may be placed throughout the statements.
 
\begin{lstlisting}
return_type | Void method_identifier ([var_type var_identifier[, var_type var_identifier[, var_type var_identifier[...]]]]){
	statement*
	[statement*]
	[statement*]
	...
}
\end{lstlisting}

Finally, we may define a method to do our pythagorean theorem calculation.

\begin{lstlisting}[label=Method Invocation,caption=Method Definition for the Pythagorean Theorem]
Float pythagorean_theorem(Float a, Float b){
	c := ( a^2 + b^2 )^0.5;
	return c;
}
\end{lstlisting}

\subsection{Class Declaration}
A class definition always starts with the keyword \verb!class! followed by an identifier. It optionally has the keyword \verb!extends! followed by the identifier of the superclass. What follows is the class body in brackets: an optional \verb!main! method and the three access-level member sections. There may be \verb!init! methods in any of the three sections.

\begin{lstlisting}
class class_identifier [extends superclass_identifier]{
	init_method
	[main_method]
	[private{
		[private_method_decl | private_var_decl]
		[private_method_decl | private_var_decl]
		...
	}]
	[protected{
		[protected_method_decl | protected_var_decl]
		[protected_method_decl | protected_var_decl]
		...
	}]
	[public {
		[public_method_decl | public_var_decl]
		[public_method_decl | public_var_decl]
		...
	}]
	[refinement {
		[refinement_decl]
		[refinement_decl]
		...
	}]
}
\end{lstlisting}

Let's make a basic geometric shape class in anticipation of later examples. We have private members, two access-level sections and an init method. No extends is specified, so it is assumed to inherit from Object.


\begin{lstlisting}[label=Method Invocation,caption=Class Declaration for a Geometric Shape class]
class Geometric_Shape{
	private{
		String name;
		Float area;
		Float circumfrence;
	}
	public{
	    init (String name){
	    	    this.name = name;
	    	    if(refinable(improve_name)){
	    	    		this.name += refine(improve_name to String)
	    	    	} 
	    	    	return
	    }
	    Float get_area(){
	    		Float area;
	    		area := refine(custom_area to Float)
	    }
	}
}
\end{lstlisting}

\subsection{Class Instantiation}
Making a new instance of a class is simple.

\begin{lstlisting}
new class_identifier([expr[,expre[,expr[...]]]])
\end{lstlisting}

For instance:

\begin{lstlisting}[label=Method Invocation,caption=Class Instantiation for a Geometric Shape class]
Geometric_Shape = new Geometric_Shape("circle");
\end{lstlisting}

Note: If literal classes are always objects, can you use the \verb!new! construct with them? The answer has to be no. It conveys no benefit even if it was usable.

\subsection{Array Initialization}
Initializing an array is almost the same as initializing a normal variable, simply add square brackets after the type.

\begin{lstlisting}
type[] identifier[, identifier[, identifier[...]]];
\end{lstlisting}

\subsection{Conditional Structures}

\subsection{Refinements}