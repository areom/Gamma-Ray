\section{Operators and Literal Types}
The following defines the approved behaviour for each combination of operator and literal type. If the literal type is not listed for a certain operator, the operator's behaviour for the literal is undefined. These operators never take operands of different types.
\subsection{The Operator {\tt =}}
\subsubsection{Integer}
If two Integer instances have the same value, \verb!=! returns \verb!true!. If they do not have the same value, it returns \verb!false!.
\subsubsection{Float}
If two Float instances have an absolute difference of less than or equal to an epsilon of $2^{-24}$, \verb!=! returns \verb!true!. If the absolute difference is greater than that epsilon, it returns \verb!false!.
\subsubsection{Boolean}
If two Boolean instances have the same keyword, either \verb!true! or \verb!false!, \verb!=! returns \verb!true!. If their keyword differs, it returns \verb!false!.
\subsubsection{String}
If two String instances have the same sequence of characters, \verb!=! returns \verb!true!. If their sequence of characters differs, it returns \verb!false!.

\subsection{The Operators {\tt =/=} and {\tt <>}}
\subsubsection{Integer}
If two Integer instances have a different value, \verb!=/=! and \verb!<>! return \verb!true!. If they do have the same value, they returns \verb!false!.
\subsubsection{Float}
If two Float instances have an absolute difference of greater than than an epsilon of $2^{-24}$, \verb!=! returns \verb!true!. If the absolute difference is less than or equal to that epsilon, it returns \verb!false!.
\subsubsection{Boolean}
If two Boolean instances have different keywords, \verb!=/=! and \verb!<>! return \verb!true!. If their keywords are the same, they return \verb!false!.
\subsubsection{String}
If two String instances have the different sequences of characters, \verb!=/=! and \verb!<>! return \verb!true!. If their sequence of characters is the same, they return \verb!false!.

\subsection{The Operator {\tt <}}
\subsubsection{Integer and float}
If the left operand is less than the right operand, \verb!<! returns \verb!true!. If the right operand is less than or equal to the left operand, it returns \verb!false!.
\subsubsection{String}
If the left operand comes before the right operand in dictionary order, \verb!<! returns \verb!true!. If the left operand comes after the right operand in dictionary order , it returns \verb!false!. If the two operands have the same sequence of characters, it returns \verb!false!.

\subsection{The Operator {\tt >}}
\subsubsection{Integer and float}
If the left operand is greater than the right operand, \verb!>! returns \verb!true!. If the right operand is greater than or equal to the left operand, it returns \verb!false!.
\subsubsection{String}
If the left operand comes after the right operand in dictionary order, \verb!<! returns \verb!true!. If the left operand comes before the right operand in dictionary order , it returns \verb!false!. If the two operands have the same sequence of characters, it returns \verb!false!.

\subsection{The Operator {\tt <=}}
\subsubsection{Integer and float}
If the left operand is less than or equal to the right operand, \verb!<! returns \verb!true!. If the right operand is less than the left operand, it returns \verb!false!.
\subsubsection{String}
If the left operand comes before the right operand in dictionary order, \verb!<! returns \verb!true!. If the left operand comes after the right operand in dictionary order , it returns \verb!false!. If the two operands have the same sequence of characters, it returns \verb!true!.

\subsection{The Operator {\tt >=}}
\subsubsection{Integer and float}
If the left operand is greater than or equal to the right operand, \verb!>! returns \verb!true!. If the right operand is greater than the left operand, it returns \verb!false!.
\subsubsection{String}
If the left operand comes after the right operand in dictionary order, \verb!<! returns \verb!true!. If the left operand comes before the right operand in dictionary order , it returns \verb!false!. If the two operands have the same sequence of characters, it returns \verb!true!.
<<<<<<< HEAD

\subsection{The Operator {\tt +}}
\subsubsection{Integer and Float}
\verb!+! returns the sum of the two operands.
\subsubsection{String}
\verb!+! returns the concatenation of the right operand onto the end of the left operand.

\subsection{The Operator {\tt -}}
\subsubsection{Integer and Float}
\verb!-! returns the right operand subtracted from the left operand.

\subsection{The Operator {\tt *}}
\subsubsection{Integer and Float}
\verb!*! returns the product of the two operands.

\subsection{The Operator {\tt /}}
\subsubsection{Integer and Float}
\verb!/! returns the left operand divided by the right operand.

\subsection{The Operator {\tt \%}}
\subsubsection{Integer and Float}
\verb!%! returns the modulo of the left operand by the right operand.

\subsection{The Operator {\tt hat}}
\subsubsection{Integer and Float}
\verb!^! returns the left operand raised to the power of the right operand.