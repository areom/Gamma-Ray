\section{Semantics}
\subsection{Types and Variables}
Every \textit{variable} in Gamma is initalized with a \textit{type} and an \textit{identifier}. The typing is static and will always be known at compile time for every variable. The variable itself holds a reference to an instance of that type. At compile time, each new instantiation reserves space for one instance of that type. To be an instance of a type, an instance must be an instance of the class of the same name as that type, an instance of one of the set of subclasses of that class, or the null instance.
\subsection{Classes, Subclasses and Their Members}
\Lang{} is a pure object-oriented language, which means almost every value here is an object, with a few exceptions like null, void and this. A class always extends another class, inherits all of its superclass' methods and may refine the methods of its superclass. A class must contain an initialization method named init and three sets of \textit{members} organized in \textit{private}, \textit{protected}, and \textit{public} sections. Members may be either variables or methods. Each class must be derived from another class as a \textit{subclass}.

\subsubsection{The Object Class}
The Object class is the superclass of all class hierarchy in \Lang{}. All objects directly or indirectly inherit from it and share its methods. By default, class declarations without extending explicitly are subclasses of Object.

\subsubsection{The Literal Classes}
There are several \textit{literal classes} that contain unique members that hold strictly data. These classes generally have methods developed for most operators. They are also all subclasses of Object.

\subsection{Methods}
A method is a reusable subdivision of code that takes multiple instances as arguments and returns either a variable of the type specified for the method or null. Methods of a class can have \textit{refine} statements placed in their bodies. Then subclasses must implement \textit{refinements}, special methods that are called in place of their superclass' refine statements. Optionally, the superclass can conditionally check if the refinement has been implemented to avoid runtime errors.

\subsubsection{Operators}
Since all variables are classes, every operator is in truth a method called from the most precident opperand with the less precident opperand as an argument. If an operator is not usable with a certain literal class, then it will not have the method implemented as a member. Classes that are not literal classes may implement those functions so as to allow that operation to be performed between two instances of the appropriate type.