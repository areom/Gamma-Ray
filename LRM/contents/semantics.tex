\section{Semantics}
\subsection{Types and Variables}
Every \textit{variable} in Gamma is declared with a \textit{type} and an \textit{identifier}. The typing is static and will always be known at compile time for every variable. The variable itself holds a reference to an instance of that type. At compile time, each new instantiation reserves space for one instance of that type. To be an instance of a type, an instance must be an instance of the class of the same name as that type, an instance of one of the set of descendants (via the subclass -- that is extends -- relation) of that class, or the {\tt null} instance.
\subsection{Methods}
A method is a reusable subdivision of code that can take multiple (possibly zero) values as arguments and either returns either a value (possibly {\tt null}) of the return type specified for the method or nothing (if the return type is {\tt void}).
\subsection{Classes, Subclasses and Their Members}
\Lang{} is a pure object-oriented language, which means almost every value here is an object -- with the exceptions that {\tt null} is the reserved reference for the lack of an object, {\tt this} is reserved to be the object of the current context (when not in a main), and {\tt void} is taken not as a value but at the lack of a return value for a method or refinement. A class always extends another class with Object being the default superclass; a class inherits all of its superclass's methods and may refine the methods of its superclass. A class must contain a constructor routine named {\tt init} and it must invoke its superclass's constructor via the super keyword -- either directly or transitively by referring to other constructors within the class. Additionally, a class contains three sets of \textit{members} organized in \textit{private}, \textit{protected}, and \textit{public} sections. Members may be either variables or methods. Each class must be derived from another class as a \textit{subclass}. Methods of a class can have \textit{refine} statements placed in their bodies. Then subclasses must implement required \textit{refinements} (refinements are required when they are not guarded by a {\tt refinable} test), special methods that are called for in place of their superclass' refine statements. Optionally, the superclass can conditionally check if the refinement has been implemented to avoid runtime errors.
\subsubsection{The Object Class}
The Object class is the superclass of all class hierarchy in \Lang{}. All objects directly or indirectly inherit from it and share its methods. By default, class declarations without extending explicitly are subclasses of Object.
\subsubsection{The Literal Classes}
There are several \textit{literal classes} that contain unique members that hold strictly data. These classes generally have methods developed for most operators. They are also all subclasses of Object.
