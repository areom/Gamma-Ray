\subsection{Examples Gamma Programs}
\subsubsection{Hello World}
This program simply prints "Hello World". It demonstrates the fundamentals needed to write a Gamma program.

\lstinputlisting[caption="Hello World in Gamma"]{../../ray/compiler-tests/programs/helloworld.gamma}

\lstinputlisting[caption="Hello World in Compiled C",language=C]{../../ray/headers/helloworld.c}

\subsubsection{I/O}
This program prompts the user for an integer and a float. It converts the integer to a float and adds the two together. It then prints the equation and result. (You might recognize this from the tutorial.)

\lstinputlisting[caption="I/O in Gamma"]{../../ray/compiler-tests/programs/io.gamma}

\lstinputlisting[caption="I/O in Compiled C",language=C]{../../ray/headers/io.c}

\subsubsection{Argument Reading}
This program prints out each argument passed to the program.

\lstinputlisting[caption="Argument Reading in Gamma"]{../../ray/compiler-tests/programs/args.gamma}

\lstinputlisting[caption="Argument Reading in Compiled C",language=C]{../../ray/headers/args.c}