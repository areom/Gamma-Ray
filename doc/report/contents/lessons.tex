
\begin{itemize}
\item {\bf Arthy}\\
First of all, I should thank my wonderful team mates and I enjoyed every bit working with them. Be it clearly silly
questions on the language or design or OCAML anything and everything they were always there! And without them it would
have certainly not been possible to have pulled this project i must confess well yea at the last moment. Thanks guys!

Thanks to Professor Edwards for making this course so much fun - you never feel the pressure of taking a theoretical course as this - as he puts it - "...in how many other theoretical courses have you had a lecture that ends with a tatooed hand.." 

As any team projects we had our own idiosyncracies that left us with missing deadlines and extending demo deadline and what not - so we were not that one off team which miraculously fit well - we were just like any other team but a team that learnt lessons quickly applied them - left ego outside the door - and worked for the fun of the project! If the team has such a spirit that's all that is required.

DOs and DONTs:
1. Have a team lead 
2. Have one person who is good in OCAML if possible or at least has had experiences with modern programming languages.
3. Have one who is good in programming language theory 
4. Ensure you have team meetings - if people do not turn up or go missing - do open up talk to them 
5. Ensure everyone is comfortable with the project and is at the same pace as yours early on
6. Discuss the design and make a combined decision - different people think differently that definitely will help.
7. This is definitely a fun course and do not spoil it by procastrination - with OCAML you just have few lines to 
code why not start early and get it done early (Smiley)
8. I may want to say do not be ambitious - but in retrospect - I learnt a lot - and may be wish some more - so 
try something cool - after all that's what is grad school for!

Good luck

\item {\bf Ben}\\

\item {\bf Matthew}\\
I had a beginning of an idea of how OOP stuff worked underneath the hood, but this really opened my eyes up to how much work was going on.

It also taught me a lot about making design decisions, and how it's never a good idea to say ``this time we'll just use strings and marker values cause we need it done sooner than later'' -- if Algebraic Data Types are available, use them. Even if it means you 
have to go back and adjust old code because of previous ideas fall out of line with new ones.

I learned how annoying the idea of a NULL value in a typed system can be when we don't give casting as an option (something we should have thought about before), and how smart python is by having methods accept and name the implicit parameter themselves. Good 
job, GvR.

\item {\bf Weiyuan}\\
First I would like to say that this is a very cool, educational and fun project. 

One thing I learned from this project is that I take modern programming languages for granted. I enjoyed many comfortable features and syntactic sugar but never realized there is so much craziness under the hood. We had a long list of ambitious goals at the beginning. Many of them had to be given up as the project went on. From parsing to code generation, I faced a lot of design decisions that I did not even know existed. I gained a much better understanding of how programming languages work and why they are designed the way they are. Also, now I have a completely refreshed view when I see posts titled "Java vs. C++" on the Internet.

Another thing I learned is that proper task division, time management and effective communication are extremely important for a team project. Doing things in parallel and communicating smoothly can save you a lot of trouble.

Finally, I learned my first functional programming language OCaml and I do like it, though I still feel it's weird sometimes.

Advice for future groups:
\begin{itemize}
  \item Start early and procrastinate less
  \item Have a team leader and communicate better
  \item Enjoy it
\end{itemize}

\end{itemize}
