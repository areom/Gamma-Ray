\pagebreak 
\subsection{Development Environment}

\subsubsection{Programming Languages}
All Gamma code is compiled by the ray compiler to an intermediary file of C (ANSI ISO C90) code which is subsequently compiled to a binary file. Lexographical scanning, semantic parsing and checking, and compilation to C is all done by custom-written code in Ocaml 4.01.

The Ocaml code is compiled using the Ocaml bytecode compiler (ocamlc), the Ocaml parser generator (ocamlyacc), and the Ocaml lexer generator (ocamllex). Incidentally, documentation of the Ocaml code for internal use is done using the Ocaml documentation generator (ocamldoc). The compilation from intermediary C to bytecode is done using the GNU project C and C++ compiler (GCC) 4.7.3.

Scripting of our Ocaml compilation and other useful command-level tasks is done through a combination of the GNU make utility (a Makefile) and the dash command interpreter (shell scripts).

\subsubsection{Development Tools}
Our development tools were minimalistic. Each team member had a code editor of choice (emacs, vim, etc.). Content management and collaboration was done via git. Our git repository was hosted on BitBucket by Atlassan Inc. The ocaml interpreter shell was used for testing purposes, as was a large suite of testing utilities written in ocaml for the task. Among these created tools were:
\begin{itemize}
\item canonical - Takes an input stream of brace-style code and outputs the whitespace-style equivalent
\item cannonize - Takes an input stream of whitespace-style code and outputs the brace-style equivalent
\item classinfo - Analyzes the defined members (methods and variables) for a given class
\item freevars - Lists the variables that remain unbound in the program
\item inspect - Stringify a given AST
\item prettify - Same as above but with formatting
\item streams - Check a whitespace-style source for formatting issues
\end{itemize}