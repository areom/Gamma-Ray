\lstset{language=caml,frame=single,
  xleftmargin=\parindent}
\subsection{Ocaml Style Guide for the Development of the Ray Compiler}

Expert Ocaml technique is not expected for the development of ray, however there are some basic stylistic tendencies that are preferred at all times.



All indentation should be increments of four spaces. Tabs and two space increment indentation are not acceptable.

\begin{lstlisting}
let x = 2
let z =
    let add5 a =
        + a 5 in
    add5 x
\end{lstlisting}



When constructing a \verb|let...in| statement, the associated in must not be alone on the final line. For a large \verb|let| statement that defines a variable, store the final operational call in a dummy variable and return that dummy. For all but the shortest right-hand sides of \verb|let| statements, the right-hand side should be placed at increased indentation on the next line.

\begin{lstlisting}
let get_x =
    ...
    let n = 2 in
    let x =
        x_functor1 (x_functor2 y z) n in
    x
\end{lstlisting}



\verb|match| statements should always include a \verb[|[ for the first item. The \verb[|[ operators that are used should have aligned indentation, as should \verb|->| operators, functors that follow such operators and comments. Exceedingly long functors should be placed at increased indentaiton on the next line. (These rules also apply to \verb|type| definitions.)

\begin{lstlisting}
let unify_it var =
    match var with
    | X(y)      ->   y                       (* pop out *)
    | Y(y) :: _ ->   to_X y                  (* convert *)
    | Z(y)      ->
        to_X (to_Y (List.hd (List.rest y))   (* mangle *)
\end{lstlisting}



All records should maintain a basic standard of alignement and indentation for readibility. (Field names, colons, and type specs should all be aligned to like.)

\begin{lstlisting}
type person = {
    names  : string list;
    job    : string option;  (* Not everybody has one *)
    family : person list;
    female : bool;
    age    : int;
}
\end{lstlisting}