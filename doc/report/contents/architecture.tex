\section{Architectural Design}
\subsection{Block Diagrams}
\subsubsection{Structure by Module}
\begin{center}
\begin{tikzpicture}[
  font=\sffamily,
  every matrix/.style={ampersand replacement=\&,column sep=2cm,row sep=2cm},
  source/.style={draw,thick,rounded corners,fill=yellow!20,inner sep=.3cm},
  process/.style={draw,thick,rounded corners,fill=blue!20},
  sink/.style={source,fill=green!20},
  dots/.style={gray,scale=2},
  to/.style={->,>=stealth',shorten >=1pt,semithick,font=\sffamily\footnotesize},
  every node/.style={align=center}]

  % Position the nodes using a matrix layout
  \matrix{
    \node[source] (source_code) {Gamma Source Code}; \& \\
    \node[process] (inspect) {Inspector}; \& \\
    \node[process] (parser) {Parser}; \& \\
    \node[process] (klass_data) {KlassData}; \& \\
    \node[process] (gen_cast) {GenCast}; \&
    \node[process] (build_sast) {BuildSast/Unanonymous};\\
    \node[process] (gen_c) {GenC}; \& \\
    \node[process] (gcc) {GCC}; \& \\
    \node[source] (csource) {Gamma Bytecode}; \& \\
  };

  % Draw the arrows between the nodes and label them.
  \draw[to] (source_code) -- node[midway,left] {a string\\of characters} (inspect);
  \draw[to] (inspect) -- node[midway,left] {a list of tokens} (parser);
  \draw[to] (parser) -- node[midway,left] {an AST} (klass_data);
  \draw[to] (klass_data) -- node[midway,left] {a class\_data\\object} (build_sast);
  \draw[to] (klass_data) -- node[midway,left] {a class\_data\\object} (gen_cast);
  \draw[to] (build_sast) -- node[midway,below] {a S-AST} (gen_cast);
  \draw[to] (gen_cast) -- node[midway,left] {a C-AST} (gen_c);
  \draw[to] (gen_c) -- node[midway,left] {a C program} (gcc);
  \draw[to] (gcc) -- node[midway,left] {bytecode} (csource);
\end{tikzpicture}
\end{center}

\subsubsection{Structure by Toplevel Ocaml Function}
\begin{center}
\begin{tikzpicture}[
  font=\sffamily,
  every matrix/.style={ampersand replacement=\&,column sep=2cm,row sep=2cm},
  source/.style={draw,thick,rounded corners,fill=yellow!20,inner sep=.3cm},
  process/.style={draw,thick,rounded corners,fill=blue!20},
  sink/.style={source,fill=green!20},
  dots/.style={gray,scale=2},
  to/.style={->,>=stealth',shorten >=1pt,semithick,font=\sffamily\footnotesize},
  every node/.style={align=center}]

  % Position the nodes using a matrix layout
  \matrix{
    \node[source] (source_code) {Gamma Source Code};\& \node[process] (inspect) {Inspector.from\_channel}; \& \node[process] (parser) {Parser.cdecl}; \\
     \& \node[process] (klass_data) {KlassData.build\_class\_data}; \& \\
     \node[process] (build_sast) {BuildSast.ast\_to\_sast}; \& \node[process] (unam) {Unanonymous.deanonymize}; \& \node[process] (gen_cast) {GenCast.sast\_to\_cast};\\
    \& \node[process] (build_sastr) {BuildSast.update\_refinements}; \& \node[process] (gen_c) {GenC.cast\_to\_c}; \\
    \& \node[source] (csource) {Gamma Bytecode}; \& \node[process] (gcc) {GCC}; \\
  };

  % Draw the arrows between the nodes and label them.
  \draw[to] (source_code) -- node[midway,above] {a sstring\\of characters} (inspect);
  \draw[to] (inspect) -- node[midway,above] {a list of tokens} (parser);
  \draw[to] (parser) -- node[midway,right] {an AST} (klass_data);
  \draw[to] (klass_data) -- node[midway,left] {a class\_data\\object} (build_sast);
  \draw[to] (klass_data) -- node[midway,left] {a class\_data\\object} (unam);
  \draw[to] (klass_data) -- node[midway,left] {a class\_data\\object} (gen_cast);
  \draw[to] (build_sast) -- node[midway,below] {a S-AST} (unam);
  \draw[to] (unam) -- node[midway,left] {a S-AST} (build_sastr);
  \draw[to] (build_sastr) -- node[midway,right] {a S-AST} (gen_cast);
  \draw[to] (gen_cast) -- node[midway,left] {a C-AST} (gen_c);
  \draw[to] (gen_c) -- node[midway,left] {a C\\program} (gcc);
  \draw[to] (gcc) -- node[midway,below] {bytecode} (csource);
\end{tikzpicture}
\end{center}

\subsection{Component Connective Interfaces}
\lstinputlisting[numbers=none,caption=The Main Ray Compiler Ocaml (Trimmed),linerange=19-67]{../../ray/ray.ml}

The primary functionality of the compiler is collected into convenient ocaml modules. From the lexer to the C-AST to C conversion, the connections are the passing of data representations of the current step to the main function of the following module. We utilize as data representations three ASTs (basic, semantic, and C-oriented), a more searchable tabulation of class data, and, of course, a source string and a list of tokens. The presence of Anonymous classes complicates the building of the array of class data and the sast as can be seen by the functor \verb!do_deanom!. Our testing experiences also lead to a more verbose form of AST generation for experimental features, hence \verb!get_data!. In all other cases, the result of the previous step is simply stored in a variable by \verb!let! and passed to the next step. The output of ray is a C file. The user must manually do the final step of compiling this file to bytecode using GCC.

\subsection{Component Authorship}
Each component was a combined effort. This is expressed somewhat in the project role section. However, for clarity, it will be reexpressed in terms of the module architecture above:

\begin{itemize}
\item Inspector - Weiyuan/Arthy
\item Parser - Ben/Arthy/Matthew
\item KlassData - Matthew
\item Unanonymous - Matthew
\item BuildSast - Matthew/Weiyuan/Arthy
\item GenCast - Matthew/Weiyuan/Ben/Arthy
\item GenC - Matthew/Weiyuan/Ben/Arthy
\item GCC - GNU
\end{itemize}
