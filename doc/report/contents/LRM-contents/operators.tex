\subsection{Operators and Literal Types}
The following defines the approved behaviour for each combination of operator and literal type. If the literal type is not listed for a certain operator, the operator's behaviour for the literal is undefined. These operators never take operands of different types.
\subsubsection{The Operator {\tt =}}
\paragraph{Integer}
If two Integer instances have the same value, \verb!=! returns \verb!true!. If they do not have the same value, it returns \verb!false!.
\paragraph{Float}
If two Float instances have an absolute difference of less than or equal to an epsilon of $2^{-24}$, \verb!=! returns \verb!true!. If the absolute difference is greater than that epsilon, it returns \verb!false!.
\paragraph{Boolean}
If two Boolean instances have the same keyword, either \verb!true! or \verb!false!, \verb!=! returns \verb!true!. If their keyword differs, it returns \verb!false!.

\subsubsection{The Operators {\tt =/=} and {\tt <>}}
\paragraph{Integer}
If two Integer instances have a different value, \verb!=/=! and \verb!<>! return \verb!true!. If they do have the same value, they returns \verb!false!.
\paragraph{Float}
If two Float instances have an absolute difference of greater than than an epsilon of $2^{-24}$, \verb!=! returns \verb!true!. If the absolute difference is less than or equal to that epsilon, it returns \verb!false!.
\paragraph{Boolean}
If two Boolean instances have different keywords, \verb!=/=! and \verb!<>! return \verb!true!. If their keywords are the same, they return \verb!false!.

\subsubsection{The Operator {\tt <}}
\paragraph{Integer and float}
If the left operand is less than the right operand, \verb!<! returns \verb!true!. If the right operand is less than or equal to the left operand, it returns \verb!false!.

\subsubsection{The Operator {\tt >}}
\paragraph{Integer and float}
If the left operand is greater than the right operand, \verb!>! returns \verb!true!. If the right operand is greater than or equal to the left operand, it returns \verb!false!.

\subsubsection{The Operator {\tt <=}}
\paragraph{Integer and float}
If the left operand is less than or equal to the right operand, \verb!<! returns \verb!true!. If the right operand is less than the left operand, it returns \verb!false!.

\subsubsection{The Operator {\tt >=}}
\paragraph{Integer and float}
If the left operand is greater than or equal to the right operand, \verb!>! returns \verb!true!. If the right operand is greater than the left operand, it returns \verb!false!.

\subsubsection{The Operator {\tt +}}
\paragraph{Integer and Float}
\verb!+! returns the sum of the two operands.

\subsubsection{The Operator {\tt -}}
\paragraph{Integer and Float}
\verb!-! returns the right operand subtracted from the left operand.

\subsubsection{The Operator {\tt *}}
\paragraph{Integer and Float}
\verb!*! returns the product of the two operands.

\subsubsection{The Operator {\tt /}}
\paragraph{Integer and Float}
\verb!/! returns the left operand divided by the right operand.

\subsubsection{The Operator {\tt \%}}
\paragraph{Integer and Float}
\verb!%! returns the modulo of the left operand by the right operand.

\subsubsection{The Operator {\tt \textasciicircum}}
\paragraph{Integer and Float}
\verb!^! returns the left operand raised to the power of the right operand.

\subsubsection{The Operator {\tt :=}}
\paragraph{Integer, Float, and Boolean}
\verb!:=! assigns the right operand to the left operand and returns the value of the the right operand. This is the sole right precedence operator.

\subsubsection{The Operators {\tt +=, -=, *=, /= \%=, and \textasciicircum=}}
\paragraph{Integer, Float, and Boolean}
This set of operators first applies the operator indicated by the first character of each operator as normal on the operands. It then assigns this value to its left operand.

\subsubsection{The Operator {\tt and}}
\paragraph{Boolean}
\verb!and! returns the conjunction of the operands.

\subsubsection{The Operator {\tt or}}
\paragraph{Boolean}
\verb!or! returns the disjunction of the operands.

\subsubsection{The Operator {\tt not}}
\paragraph{Boolean}
\verb!not! returns the negation of the operands.

\subsubsection{The Operator {\tt nand}}
\paragraph{Boolean}
\verb!nand! returns the negation of the conjunction of the operands.

\subsubsection{The Operator {\tt nor}}
\paragraph{Boolean}
\verb!nor! returns the negation of the disjunction of the operands.

\subsubsection{The Operator {\tt xor}}
\paragraph{Boolean}
\verb!xor! returns the exclusive disjunction of the operands.

\subsubsection{The Operator {\tt refinable}}
\paragraph{Boolean}
\verb!refinable! returns \verb!true! if the refinement is implemented in the current subclass. It returns \verb!false! otherwise.