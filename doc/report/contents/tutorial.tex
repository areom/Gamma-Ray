\section{Language Tutorial}

The structure of the example below should be intimately familiar to any student of Object-Oriented Programming. 
\lstinputlisting[caption="A simple I/O example"]{../../ray/compiler-tests/programs/io.gamma}

We start with a definition of our class.

\begin{lstlisting}
class IOTest:
\end{lstlisting}

We follow by starting a \verb!public! access level, defining an \verb!init! method for our class, and calling the \verb!super! method inside the init method. (Since we have not indicated a superclass for \verb!IOTest!, this \verb!super! method is for \verb!Object!.)

\begin{lstlisting}
    public:
        init():
            super()
\end{lstlisting}

We also define the \verb!private! access level with three methods: a generic method that prints a prompt message and two prompts for \verb!Integers! and \verb!Floats! respectively. These prompts call the generic message and then read from \verb!system.in!.

\begin{lstlisting}
  private:
    void prompt(String msg):
      system.out.printString(msg)
      system.out.printString(": ")

    Integer promptInteger(String msg):
      prompt(msg)
      return system.in.scanInteger()

    Float promptFloat(String msg):
      prompt(msg)
      return system.in.scanFloat()
\end{lstlisting}

We then write a method under the \verb!public! access level. This calls our \verb!private! level methods, convert our \verb!Integer! to a \verb!Float! and print our operation.

\begin{lstlisting}
    void interact():
      Printer p := system.out
      Integer i := promptInteger("Please enter an integer")
      Float f := promptFloat("Please enter a float")
      p.printString("Sum of integer + float = ")
      p.printFloat(i.toF() + f)
      p.printString("\n")
\end{lstlisting}

Finally, we define the \verb!main! method for our class. We just make a new object of our class in that method and call our sole public method on it.


\begin{lstlisting}
  main(System system, String[] args):
    IOTest test := new IOTest()
    test.interact()
\end{lstlisting}